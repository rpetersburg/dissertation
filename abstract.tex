% Not to exceed 750 words
\begin{abstract}
The current generation of radial velocity spectrographs are at the precipice of discovering a multitude Earth-like exoplanets orbiting in the habitable zones of nearby stars. Such detections require Doppler precision of approximately 10\cms, an order of magnitude better than the typical best-case measurement from the previous generation of instruments. Therefore, the radial-velocity community requires research and innovation from all angles to push our technology over the brink. This thesis present multiple contributions to this field---ranging from the development of precision laser equipment to implementation of advanced statistical data analysis algorithms---all in support of the EXtreme PREcision Spectrograph with the goal of improving instrument precision and exoplanet detection capability.

In Chapter \ref{chapter:modal-noise}, we demonstrate the effectiveness of quasi-chaotic high-amplitude agitation as an optimal form of modal noise mitigation in the optical fibers that feed into radial-velocity spectrographs. This technique was shown to improve radial-velocity error for a single-wavelength laser line from more than 10\ms to less than 60\cms without affecting focal ratio degradation within the fiber. After development of an agitator based on this method for use with EXPRES, we found that combined radial-velocity precision across an entire laser frequency comb improved from 32.8\cms to 6.6\cms.

In Chapter \ref{chapter:astro-comb}, I present aluminum nitride as a non-linear optical material that can support frequency comb development from near-infrared to ultraviolet wavelengths. By injecting light from an aluminum nitride micro-ring into EXPRES, I demonstrated the material's viability of producing resolvable comb lines throughout the bandpass of the instrument. I also prototyped a 16\si{\giga\hertz} electro-optic modulation comb in combination with an aluminum nitride waveguide as a device that could produce a cheaper and wider bandwidth visible-wavelength laser frequency comb for radial-velocity spectrograph wavelength calibration.

Finally, in Chapters \ref{chapter:pipeline} and \ref{chapter:pipeline2}, I present the EXPRES data extraction pipeline and the numerous novel algorithms that went into its design. Through the default version of the pipeline, including a flat-relative optimal extraction and chunk-by-chunk forward model radial-velocity measurement, we have achieved 30\cms single-measurement precision on observations with a signal-to-noise ratio of 250 measured at 550~\si{\nano\meter}. As demonstrated with 51 Peg b, the residual scatter of these observations after fitting with a single-planet Keplerian orbit is less than 90\cms. As alternatives to the default techniques, I also present my implementations of flat-relative spectro-perfectionism and \textit{B}-spline regression stellar templates within the EXPRES pipeline. These methods provide comparable radial-velocity precision on observations of HD 3651 while also opening up many possibilities for future explorations with radial-velocity data analysis.
\end{abstract}