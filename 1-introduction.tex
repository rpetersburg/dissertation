\chapter{Introduction} \label{intro}

Twenty-five years ago, the Nobel Prize-winning discovery of an \textit{exoplanet}, a planet outside of our solar system, orbiting a Sun-like star shook the astronomical community and amplified a wave of exoplanet science that thrives to this day. \citet{mayor_jupiter-mass_1995} detected the planet indirectly by measuring the motion of the planet's host star over time, rather than the planet itself, using what is called the \textit{radial-velocity technique}. These measurements required a \textit{high-resolution spectrograph}---an instrument that can measure the electromagnetic spectrum of the star to high precision---taking stellar light, coupled via optical fiber, from a telescope at the Haute-Provence Observatory in France \citep{baranne_elodie_1996}.

With this technique and instrument, they were able to detect 51 Pegasi b, a planet about 150 times more massive than Earth with an orbit closer than that of Mercury. It is quite a testament that, more than two decades later, we are at the breaking point of using the radial-velocity technique with ground-based fiber-fed spectrographs---essentially the exact same methodology and technology---to discover Earth-mass planets orbiting within their host star's habitable zone.

In this thesis, I describe my personal contributions towards the technological advancements necessary to detect exoplanets at such extreme precision. These advancements encompass multiple approaches to technology development, from instrument design and the mitigation of physical noise sources to the implementation of novel data analysis algorithms through software. Such a holistic approach towards instrumentation has been crucial in clarifying the steps required to reach the next generation of exoplanet measurement. Through this introduction, I aim to demystify the language used to describe the systems studied in my research and reveal some links between them.

\section{The Radial-velocity Technique} \label{intro:eprv}

The principles of the radial-velocity technique are built upon two fundamental concepts in physics: Newton's Third Law of Motion and the Doppler Effect. As a planet orbits around its host star, naturally affected by the gravitational pull of the star, the planet itself imparts an equal but opposite force on the star. Thus, each object in this two-body system follows its own orbit around the center of mass of the system. If we can measure orbit-like periodicity in the motion of the star, we can infer the existence of an exoplanet. Importantly, the motion we choose to measure is the stellar \textit{radial} velocity---movement directly towards and away from us, the observer---rather than any \textit{transverse} velocity (up, down, left, or right), which we leave to the field of astrometry \citep[e.g.][]{} [CITATIONS HERE PLEASE].

But how does one actually measure a stellar radial velocity (RV)? Using the Doppler Effect! As light is emitted from the star (and before it takes a multi-year journey to Earth), the relative motion of the star causes the physical characteristics of this light to change depending on how the star is moving. If the star is moving towards Earth (with negative velocity $v$), the frequency ($f$) of the light increases (equivalent to saying the wavelength $\lambda$ of the light decreases) and is therefore \textit{blue-shifted} as higher frequency light appears bluer to the human eye. If the star is moving away from Earth (positive $v$), the light is oppositely \textit{red-shifted} ($f$ decreases, $\lambda$ increases). These shifts in frequency can be subsequently converted into relative RV measurements of the star using the relativistic Doppler relation
\begin{equation}
    \frac{f_s}{f_o} = \frac{\lambda_o}{\lambda_s} = \sqrt{\frac{c - v}{c + v}}
\end{equation}
where $o$ and $s$ represent measurement at the observer and star respectively, while $c$ is the speed of light in a vacuum. Therefore, the more precise goal of using the RV technique is to measure light from the star at many points in time, calculate the blue/red-shift of the light at each time, and convert these shifts into RVs.

This leaves us now with two questions:
\begin{enumerate}
    \item How do we ``measure light'' from a star?
    \item What can we actually do with these stellar RVs?
\end{enumerate} I'll start by answering the second of these questions first.

\subsection{The Keplerian RV Model}

Time series of stellar RVs can be used to infer parameters that describe a planet and its orbit around the host star. These parameters include, most critically, the mass of the planet and its orbital period, the time it takes to complete one full orbit. However, it takes a bit of physics and math to figure out where this information comes from.

I'll start by describing the elements of a closed Keplerian orbit, typically written as
\begin{equation}
    r(\theta) = \frac{a (1-e^2)}{1 + e \cos{\theta}}.
    \label{eq:kepler}
\end{equation}
At any given angle $\theta$ within an orbit---also known as the true anomaly from periapsis, the position of closest approach to the center of mass of the system---the distance $r$ from the center of mass can be described as an ellipse with semi-major axis $a$ and eccentricity $e$. An orbit with $e=0$ is completely circular, since it would mean $r(\theta)=a$.

RV measurement is limited to velocity, rather than positional, data and we are stuck viewing the orbit from an arbitrary (and typically unknown) angle. Therefore, we rewrite the Keplerian model as
\begin{equation}
    v(t) = K (\cos{(\theta(t)+\omega)} + e \cos{\omega})
    \label{eq:kepler-rv}
\end{equation}
where $\omega$ is the argument of periapsis---a constant angle that describes the orientation with which we are viewing the orbit based on the position at periapsis---and $K$ is what we call the RV semi-amplitude \citep{lovis_radial_2011}.

You may notice that the true anomaly $\theta(t)$ is now represented as a function in time ($t$). Since RV data is taken through a series of observations at different times, time is the natural independent variable in our model. Through a couple of steps, we can map out the relationship between $\theta$ and $t$:
\begin{equation}
    \theta(t) = 2 \arctan{\left(\sqrt{\frac{1+e}{1-e}}\tan{\left(\frac{E(t)}{2}\right)}\right)}
    \label{eq:mean-anomaly}
\end{equation}
where the eccentric anomaly $E(t)$ (an intermediate way of describing the orbital angle) has a nonlinear relationship to the times of observation:
\begin{equation}
    E(t) - e \sin{E(t)} = \frac{2\pi (t - \tau)}{P}.
    \label{eq:eccentric-anomaly}
\end{equation}
$\tau$ is the time of pariapsis, or the time at which the orbit sits at the angle $\omega$, and $P$ is the period of the orbit.

Along with the time-dependent angle $\theta(t)$, the other important aspect of Equation \ref{eq:kepler-rv} is the constant RV semi-amplitude $K$. This value is the maximum RV of the object as it moves towards and away from the observer and can be rewritten in terms of other orbital parameters:
\begin{equation}
    K = \frac{2\pi a \sin{i}}{P\sqrt{1-e^2}}
    \label{eq:rv-semi-amplitude}
\end{equation}
where $i$ relates to the inclination of the orbit from the observer's perspective. Note that a value of $i=90^\mathrm{o}$ means that the orbit is completely parallel to our line of sight, while a value of $i=0^\mathrm{o}$ means that all orbital motion is perpendicular to our line of sight and the measured RV is always zero.

At this point, it's important to be reminded that all of parameters used in these equations are used to model the orbit of the star and \textit{not} the orbit of the planet. Our Doppler measurements, and thus velocities, all come from stellar motion. Thankfully, most of these parameters identically describe the orbit of the planet, particularly $\theta(t)$, $\omega$, $e$, $P$, and $\tau$. Moreover, the semi-major axis of the stellar orbit ($a_*$) can be converted to that of the planetary orbit ($a_P$) with knowledge of the planetary and stellar masses ($M_P$ and $M_*$ respectively) through
\begin{equation}
    a_P M_P = a_* M_*,
    \label{eq:center-of-mass}
\end{equation}
meaning the total distance between the star and planet is $a_* + a_P$. The period and semi-major axis of the planet can also be related using Kepler's Third Law:
\begin{equation}
    P^2 = \frac{4\pi^2}{G (M_P+M_*)} a_P^3
    \label{eq:kepler-third}
\end{equation}
where G is the gravitational constant.

Putting together Equations \ref{eq:rv-semi-amplitude}, \ref{eq:center-of-mass}, and \ref{eq:kepler-third}, and making the approximation $M_P + M_* \approx M_*$ (due to the fact that stars are typically much more massive than their orbiting planets) we find that 
\begin{equation}
    K = \left( \frac{2\pi G}{P M_*^2} \right)^\frac{1}{3} \frac{M_P \sin{i}}{\sqrt{1-e^2}}.
    \label{eq:rv-semi-amplitude2}
\end{equation}
Thus, by taking a series of RV data from a star and fitting it to the Keplerian model in Equation \ref{eq:kepler-rv}, we are able to measure an orbiting exoplanet's mass, period, and eccentricity---a great deal of information from this one technique!

It bears emphasizing that $K$ is a measurement of velocity that is directly proportional to planetary mass and inversely scales with the distance between the planet and the star. Within the RV community and throughout this dissertation, velocities are used as \textit{the} metric scale for determining measurement precision (also called Doppler precision) and to compare different approaches to the RV technique. For perspective, here are a few points of reference:
\begin{itemize}
    \item Earth imparts a 0.09\ms and Jupiter imparts a 12.5\ms RV semi-amplitude on the Sun. Note that 30\ms is about the speed limit on most American highways.
    \item The RV semi-amplitude measured for 51 Pegasi b, the first planet found with the RV technique, is 55.65\ms, about 600 times that of Earth.
    \item The current generation of RV instruments are pushing better than 1.0\ms RV semi-amplitude measurement precision \citep{fischer_state_2016}. This is a very slow human walking pace.
\end{itemize}

There are, of course, a few caveats about the Keplerian model when used with the RV technique. As shown in Equation \ref{eq:rv-semi-amplitude2}, the mass of the star ($M_*$) is required to make any predictions about the mass of the planet ($M_P$) and must be measured elsewhere. Furthermore, Equation \ref{eq:rv-semi-amplitude2} demonstrates that we can only put a lower limit on the mass of the planet ($M_P\sin{i}$), since the inclination of the orbit is unknown. Also, Equation \ref{eq:eccentric-anomaly} is not analytically solvable for $E(t)$, meaning it needs to be approximated numerically using an iterative root-finding algorithm such as the Newton-Rhapson or Halley method. Finally, Equation \ref{eq:kepler-rv} can be expanded for multiple planet systems by simply summing the RV contributions of each planet together to create a combined Keplerian model. However, this is merely an approximation since the gravitational pull of each planet would affect the others. A more complicated n-body Newtonian model could be used \citep[e.g.][]{rivera_75_2005, fischer_five_2008}, but the Keplerian model is sufficient for the explorations in this dissertation.

\subsection{RV Spectroscopy}

In order to study stellar light and convert it into RVs, we first need to disperse it into a spectrum, a measurement of the light's intensity at multiple specific wavelengths. This is done through a process called spectroscopy and with instruments called spectrographs. Typically, spectrographs are designed to convert spectral intensity information ("How much red light do I have?") into spatial intensity information ("Where is the red light on this image and how bright is it?"). A very basic example of a spectrograph is a prism that projects light onto a piece of paper: white light (all colors mixed together evenly) shining into the prism is dispersed into a full rainbow with blue appearing on the left side and red on the right side of the paper.

How precisely we are able to measure the contributions of given wavelengths to the spectral picture is called the resolution of the spectrograph, or
\begin{equation}
    R = \frac{\lambda}{\Delta \lambda}
\end{equation}
where $\Delta \lambda$ is the width of a measurable spectral bin. For example, a resolution 100 spectrograph would be able to measure the difference in intensity at wavelengths of 500~nm and 505~nm (two very close shades of green light) since $\frac{500}{505-500} = 100$. High-resolution spectrographs built to measure RVs are typically designed with resolutions greater than 50,000.

Spectrographs are also defined by the spectral bandwidth they can measure. Our example of the prism + paper spectrograph would have a bandwidth from 380~nm (blue) to 750~nm (red) since this is the portion of the electromagnetic spectrum visible to the human eye. The choice of bandwidth for an RV spectrograph depends on the types of stars expected to be observed. Visible-wavelength spectrographs (like our prism spectrograph) are built to observe F-, G-, and K-type stars, similar to our own sun, since these stellar spectra are brightest near yellow wavelengths. Near-infrared spectrographs (780--2500nm), on the other hand, are meant to observe the cooler (and less noisy) M-type stars.

Measurement of the Doppler Effect, and therefore RVs, using spectroscopy leverages the existence of atomic and molecular absorption lines within the stellar spectra. 

\subsection{Limitations and Alternatives}

\section{The State of the Field}

The history of the RV technique can be traced back through the various spectrographs used throughout the years to make these measurements. 


\section{The EXtreme PREcision Spectrograph} \label{intro:expres}

The majority of the work presented in this thesis was completed in support of the EXtreme PREcision Spectrograph \citep{jurgenson_expres_2016, blackman_performance_2020, petersburg_extreme-precision_2020}---a fiber-fed, high-resolution, laser-frequency-comb-calibrated spectrograph commissioned at the Lowell Discovery Telescope in 2018. Therefore, I will be using it as a guide to understand the physical and virtual mechanisms required make such precise measurements as the 1\ms wobble of a star.

Determining where to focus efforts in improving the precision of RV measurement requires a full understanding of the measuring instrument. In the case of RV spectroscopy, and in particular this thesis, this involves knowledge of four critical areas: (1) the optical fiber infrastructure, (2) echelle spectrograph design, (3) wavelength calibration sources, and (4) data extraction software. In this section, I introduce these topics as primers before even further discussion in the meat of this thesis.

\subsection{Optical Fiber Infrastructure} \label{intro:optics:fiber}



Combining multiple signals from various sources into the same instrument. Modular.

Etendue and coupling losses

Focal ratio degradation

Scrambling gain

Modal noise

\subsection{Echelle Spectroscopy} \label{intro:optics:echelle}

Fiber injection

Collimation and echelle grating dispersion

Re-collimation and prism cross-dispersion

Camera barrel re-focus and CCD

Echellogram structure

Resolution and velocity precision

\subsection{Wavelength Calibration} \label{intro:wvln_cal}

In order to extract stellar spectra, RV spectrographs use the spectra of well-characterized and stable light sources as a simultaneous or observation-bracketing reference on the spectrograph camera. Historically, thorium argon (Th-Ar) lamps and iodine reference cells have been used to calibrate RV spectrographs since their spectral properties are well understood and their bandwidth covers most of the visible spectrum. However, both of these light sources have inherent issues that limit RV measurements to approximately \SI{1}{\meter\per\second} precision. Th-Ar emission lines are broad, saturated, and irregularly spaced, meaning some wavelength regions are less well calibrated than others. The iodine technique, which applies absorption wavelengths directly on the stellar spectrum, masks the subtle spectral line profile effects imprinted by stellar activity. Furthermore, iodine introduces complexity, such as parameter cross talk, to the forward modeling \citep{spronck_fiber_2015} and the cells themselves may not have long term mechanical stability \citep{fischer_twenty-five_2014}.

More recently, RV spectrographs have employed frequency combs (FCs), synthesized spectra containing sharp peaks of intensity at equally spaced frequencies, with the intention of better stability and therefore precision in their wavelength calibration. Ideally, the lines of a FC are non-overlapping and located at precisely determined wavelengths with flat intensity, to avoid over- or under-saturating pixels on the spectrograph detector. The FC must also have the proper free spectral range (FSR, frequency separation of comb lines) across the entire bandwidth of the spectrograph optics so that the detector can resolve each peak (more than $\sim$\SI{10}{\giga\hertz}) and calibrate a sufficient number of stellar frequencies (less than $\sim$\SI{40}{\giga\hertz}). Also, the zero-point frequency offset ($f_0$) and FSR should not drift over both short (seconds) and long (months) time scales. Most FC devices have been developed for the near-infrared, due to the proliferation of telecom interest around \SI{1550}{\nano\meter}, and this has been sufficient for observing M dwarf exoplanets \citep{fischer_state_2016}. However, to address planetary system statistics for late F, G, and early K type stars, the wavelength range will need to stretch through the visible to better calibrate many more absorption lines. It is especially important to reach the Calcium H \& K lines located below \SI{400}{\nano\meter} that contain chromospheric activity information that characterizes photospheric noise \citep{isaacson_chromospheric_2010, lovis_harps_2011}.

A relatively cheap way to produce a broadband optical FC is using a tunable Fabry-P\'erot (FP), a resonant cavity that employs feedback to correct for drifts in length. They do not offer an inherent $f_0$ calibration, however, and must be referenced against a separate stable source for bootstrapped calibration \citep{mccracken_single-lock_2014, sturmer_rubidium-traced_2018} meaning the system cannot be self-contained. Therefore, fixed-length FP etalons were developed to mitigate this issue. However, \citet{reiners_laser-lock_2014} and \citet{wildi_passive_2012} have shown that the set distance between the two reflective surfaces still drifts unpredictably, especially over long time scales, thereby continuing to limit spectrograph precision to only $\sim$\SI{1}{\meter\per\second}.

Menlo Systems has built perhaps the most advanced wavelength calibrator for visible RV spectroscopy, a laser FC that reaches $\sim$\SI{1}{\centi\meter\per\second} RV precision \citep{probst_laser_2014}. This FC pulses a femtosecond mode-locked laser to produce high finesse lines and an extremely stable FSR. Unfortunately, these lines are too tightly spaced for RV spectrographs, therefore the system must use line-by-line spectral filtering with multiple tunable FP cavities to suppress most of the produced frequencies. There are consequently some critical limitations to this device: complexity, cost, and limited bandwidth. The Menlo FC requires a continuing service contract to properly maintain it and thus has not yet reach true ``turn-key'' status. Also, this technology costs approximately \$1,000,000---a prohibitive price point for smaller RV projects---and is extremely bulky, limiting its use to larger ground-based spectrographs. Furthermore, the system is still limited to $\sim$\SI{450}{\nano\meter} thereby excluding the critical Calcium H \& K lines below \SI{400}{\nano\meter}.

A promising and growing field of astro-comb development includes electro-optic modulation (EOM) combs and chip-based waveguide technologies. These devices offer ease-of-use along with a much more affordable price point [CITATIONS OUT THE WAZOO].

Emission line wavelength calibration sources have another important function for the instrument: point spread function characterization. Since an echelle spectrograph converts spectral information into spatial information, a spectral point source (i.e. a single wavelength laser) would ideally be mapped as a physical point source on the spectrograph's detector. However, Within any optical re-imaging system, light is not perfectly translated from input to output. Rather, the light undergoes optical aberrations modifying it as it is reflected and diffracted throughout the system and, importantly, these effects are typically wavelength dependent. Therefore, shining a single wavelength laser into a spectrograph reveals exactly the optical aberrations, or point spread function, of the instrument at that wavelength. Emission line wavelength calibration sources are simply a collection of hundreds or thousands of such single wavelength lasers, thus they can be used to map the point spread function of a spectrograph across its entire spectral format, as I demonstrate in Chapters \ref{chapter:astro-comb} and \ref{chapter:spec-perf}.

Finally, consideration needs to be made as to \textit{when} wavelength calibrations are made as part of the observation strategy. Namely, the decision between simultaneous calibrations---measuring the wavelength calibrator at the same time as the star except through a separate fiber---and bracketed calibrations---using the same fiber but alternating between stellar and calibration observations. Simultaneous calibrations have two inherent downsides: (1) a requirement to tailor the brightness of the calibrating light source to match signal-to-noise over various length of science observations (i.e. brighter and dimmer stars) and (2) systematic changes in the optical system may not perfectly correlate between separate fibers, meaning extra analysis is necessary to translate wavelength solutions generated by the simultaneous calibration light path to the science light path. Therefore, EXPRES uses bracketed calibrations in order to more consistently control the brightness of its laser frequency comb (which can reach a sufficient signal-to-noise level within only 10 seconds) and to calibrate the instrument using the same optical path and detector pixels as science observations. As demonstrated in Chapter \ref{chapter:pipeline}, various methods of interpolation are more than sufficient to then project these bracketed wavelength solutions to midpoint time of intermediate observations.

\subsection{Non-instrumental effects}

Telluric contamination

Stellar activity

\subsection{Data Extraction} \label{intro:extraction}

Raw Image reduction

Spectral extraction
- Boxcar
- Optimal
- Spectro-perfectionism

Wavelength calibration

Barycentric correction

Telluric modeling

Radial velocity measurement
- Cross-correlation
- Forward modelling

Stellar activity indicators

\section{Thesis Outline} \label{intro:structure}

In this thesis, I present the design and implementation of multiple novel methods---within instrumentation and data extraction---that have shown demonstrable improvement to many of the aspects of fiber-fed radial velocity spectroscopy outlined in the previous section. My work has entailed 

This thesis is split into four chapters: two (Chapters \ref{chapter:modal-noise} and \ref{chapter:pipeline}) are reprints of peer-reviewed journal articles and the other two (Chapters \ref{chapter:astro-comb} and \ref{chapter:spec-perf}) are summaries of other work that I have completed that either unfit for publication or simply fell short.