\chapter{Conclusions}\label{chapter:conclusion}

In summary, this thesis presented new research to the field of radial-velocity spectroscopy in support of instrumentation development and implementation across various regimes of EXPRES, including its fiber architecture, wavelength calibration sources, spectral extraction, and data analysis. In Chapter \ref{chapter:modal-noise}, I found that quasi-chaotic agitation, through the use of dual rotating arms, would optimally mitigate modal noise within the optical fibers leading to the spectrograph, decreasing laser frequency comb wavelength calibration velocity scatter from 32.8~\si{\centi\meter\per\second} to 6.6~\si{\centi\meter\per\second}. In Chapter \ref{chapter:astro-comb}, I demonstrated the viability of aluminum nitride as a candidate for future astro-comb development, due to its high transparency and frequency conversation efficiency across a wider band than that provided by the EXPRES laser frequency comb. In Chapter \ref{chapter:pipeline}, I introduced the default EXPRES flat-relative optimal extraction pipeline, yielding single measurement precision of less than 30~\si{\centi\meter\per\second} for observations of 51 Pegasi with a signal-to-noise greater than 250 at 550~\si{\nano\meter} and kicking off the next-generation of radial-velocity measurement with sub-\si{\meter\per\second} RV precision. Finally, in Chapter \ref{chapter:pipeline2}, I described further improvements made to the EXPRES pipeline---flat-relative spectro-perfectionism and \textit{B}-spline regression stellar templating---that provide radial-velocity results similar to those from the optimal extraction while also enabling greater diagnostic capability with our data.

In lieu of further reiteration of the conclusions already presented in the individual chapters of this thesis, I have instead organized these final conclusions into a set of lessons learned (Chapter \ref{conclusion:lessons}), a compilation of future work that could build off of this research (Chapter \ref{conclusion:future}), and my final thoughts (Chapter \ref{conclusion:final}). As a result of completing all of this work and synthesizing it into this massive document, I have been granted an extensive range of experiences that have shaped my perception of radial-velocity spectroscopy. I hope that by compiling just some of these ideas here, they can more easily be passed to those who wish to continue improving instrumentation and data extraction within this exciting field.

\section{Lessons Learned} \label{conclusion:lessons}

\textbf{Optical fiber break protection is best achieved with flexible stainless steel sheaths}, followed closely by soft rubber jackets. While testing a variety of optical fibers during the modal noise study of Chapter \ref{chapter:modal-noise}, I found that fibers jacketed with hard plastic were much more likely to break than those with metal or soft rubber sheaths. This is likely due to a certain ``breaking point'' of the hard jacket, where once it is bent tighter than a certain radius, the jacket kinks and snaps the fiber. With softer or interwoven metal sheaths, the allowed radii of curvature are admittedly much smaller---potentially leading to less transmission, especially in bluer wavelengths. But having to straighten a few tight bends throughout the fiber run is a significantly better problem to have than needing to replace an entire fiber connection between the telescope and spectrograph.

\textbf{Building an in-house electro-optic modulation comb requires a high-energy pulse-width measuring device.} The greatest problem we faced while prototyping the electro-optic modulation comb in Chapter \ref{chapter:astro-comb} was a lack of understanding the pulse width at each stage of the device. The pulse-width measurement made for Figure \ref{fig:eom-pulse} was taken with a FROG, but involved burning multiple neutral density filters due to the very high peak energy of the pulse. Investment in a more robust device early on would have certainly made this process simpler, especially since the ultimate challenge of our comb was likely insufficient pulse compression.

\textbf{Finding and fixing major outliers is more critical than incrementally tweaking algorithms.} I'll explain this using an example. Recently, in the midst of trying to improve our wavelength calibration algorithm by a few \si{\centi\meter\per\second}, we found that radial velocities from a single night were off from the rest by multiple \si{\meter\per\second}! This was caused by a failure in our interpolation scheme (a cubic polynomial fit) when less than three sets of LFCs were taken throughout the night. We fixed it by instead implementing a linear interpolation scheme with nearest neighbors weighting, decreasing velocity scatter by $\sim$1~\si{\meter\per\second} for a few target stars. Therefore, our problems were better solved by focusing on completely changing our method rather than tweaking insignificant default parameters.

\textbf{Centralizing data analysis code and nightly anomaly reporting enables quicker repairs and validation.} Data extraction and analysis for a high-resolution spectrograph is a massive enterprise. It would have not been possible to conduct much of this work by passing data from person to person in order to individually run reduction, extraction, wavelength calibration, etc. Using tools like git and features of GitHub like issue-tracking and release logs has been a life-saver, enabling better documentation and collaboration when things are going wrong. Unfortunately, one area that probably required some improvement would be central reporting of all hardware changes. With a clear record of \textit{when} the instrument was noticeably altered, we can better correlate changes in analysis results with changes in hardware.

\section{Future Work} \label{conclusion:future}

\textbf{Further electro-optic modulation comb development.} \citet{obrzud_visible_2019} used an electro-optic comb design similar to the one introduced in Chapter \ref{chapter:astro-comb} with some spectacular results. I believe we were on the right track with our design, but simply did not employ a long enough highly nonlinear fiber and therefore induced insufficient spectral broadening before the aluminum nitride waveguide. Other possible areas of exploration include implementing a chirped-fiber-Bragg grating for better dispersion control, using longer aluminum nitride waveguides, or coupling aluminum-nitride and lithium-niobate micro-rings to the electro-optic modulation comb.

\textbf{Charge Transfer Inefficiency.} One element completely missing from the EXPRES pipeline is any correction for charge transfer inefficiency \citep{goudfrooij_empirical_2006, bouchy_charge_2009, blake_impact_2017}. The mechanisms for simulating it are fairly well-known, but there has yet to be a simple pixel-wise correction model that can be applied to the reduced data. Although this effect is likely minimized on EXPRES due to consistency in signal-to-noise \citep{blackman_performance_2020}, it may play a surprisingly important (and currently uncorrected) role in laser frequency comb calibration due to the high contrast of individual comb lines.

\textbf{Telluric modeling and expanding the radial-velocity window.} I find Figure \ref{fig:chunk-vels} rather telling about where to focus radial-velocity efforts: improve telluric modeling from 550--830~\si{\nano\meter} (since the scatter here is well above 100\ms) and focus more radial-velocity analysis in the region from 400~\si{\nano\meter} to 500~\si{\nano\meter}. There is so much high fidelity data in the bluer regions of the detector and it would be a shame for us to continue not using it. As a first step towards better telluric modeling, I would recommend applying the \textit{B}-spline regression method of Chapter \ref{chapter:pipeline2} to the empirical analysis of B-stars from SELENITE, which currently uses simple order-wise interpolation methods. Naturally, this should be combined with continued improvements in suppressing stellar activity signals, but it seems to me that tellurics are underappreciated in our analyses.

\textbf{Spectro-perfectionism sampling study.} The biggest gain from using spectro-per\-fectionism over optimal extraction is its ability to sample the detector with an arbitrary bin size irrespective of the size of the pixels. I would recommend implementing flat-relative spectro-perfectionism such that it is normalized regardless of the spectral binning and then conducting a study to see how matching this sampling to instrument dispersion may improve the fidelity of extracted spectra. In particular, I would be curious to see if overlapping orders are better aligned than when using the flat-relative optimal extraction.

\textbf{Complete spectrograph forward modeling.} According to the most recent NASA Exo\-planet Exploration Analysis Program Group meeting, a high priority within the community is generating extraction code that is able to completely forward model a radial velocity into the two-dimensional pixel space of a reduced detector exposure. This would eliminate the need to move step-by-step between extraction, wavelength calibration, bary\-centric correction, telluric modeling, etc. where uncertainty has to be modeled and propagated through each step. Naturally, spectro-perfectionism and stellar templating may play a role in such analysis. Regardless, it will be fascinating to see if such a major shift in extraction techniques is even possible.

\section{Final thoughts} \label{conclusion:final}

The goal of discovering exoplanets that impart less than a 10~\si{\centi\meter\per\second} stellar radial-velocity semi-amplitude is certainly a lofty one. As demonstrated in this thesis, the EXtreme PREcision Spectrograph is definitely living up to its expectations of being a next-generation radial-velocity instrument. Getting to this point, however, required the expertise of multiple generations of spectrographs being passed down through their future iterations. This is how we have transitioned from boxcar extraction to optimal extraction to spectro-perfectionism and from iodine cells to thorium-argon lamps to mode-locked laser frequency combs to on-chip micro-ring resonators. We are absolutely within an exciting era of extreme-precision spectroscopy, finally leveraging the radial-velocity technique to detect Earth-like exoplanets and potentially discovering if there really is life somewhere out there.